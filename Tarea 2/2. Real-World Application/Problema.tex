\documentclass{article}
\usepackage{amsmath}
\usepackage{graphicx}

\title{Problema de consumo energetico de una cinta transportadora}
\author{Juan E Osorno D.}
\date{January 2025}

\begin{document}

\section{Contexto del problema precios y gastos promocionales de un shampoo}

Una empresa de elementos para el aseo personal tienen un nuevo prodcuto de shampoo "Walgreens Shampoo + plus
(16 oz)". El departamento de precios y marketing de la empresa desea conocer el precio y los gastos promocionales
del nuevo producto que desean lanzar al mercado que ayuden a incrementar sus ventas en el mercado, para obtener 
resultados mas fiables la empresa selecciona 35 ejemplos de tiendas donde venden su producto alrededor de todo 
el pais en dolares, las cuales se pueden visualizaren la siguiente tabla:

\begin{table}
    \centering
    \begin{tabular}{|c|c|c|c|}
    \hline
    \textbf{Tienda} & \textbf{Ventas} & \textbf{Precios} & \textbf{Promocion}
    1 & 4141 & 59 & 200 \\ \hline
    2 & 3842 & 59 & 200 \\ \hline
    3 & 3056 & 59 & 200 \\ \hline
    4 & 3519 & 59 & 200 \\ \hline
    5 & 4226 & 59 & 400 \\ \hline
    6 & 4630 & 59 & 400 \\ \hline
    7 & 3507 & 59 & 400 \\ \hline
    8 & 3754 & 59 & 400 \\ \hline
    9 & 5000 & 59 & 600 \\ \hline
    10 & 5120 & 59 & 600 \\ \hline
    11 & 5011 & 59 & 600 \\ \hline
    12 & 5015 & 59 & 600 \\ \hline
    13 & 1916 & 59 & 600 \\ \hline
    14 & 675 & 79 & 200 \\ \hline
    15 & 3636 & 79 & 200 \\ \hline
    16 & 3224 & 79 & 200 \\ \hline
    17 & 2295 & 79 & 200 \\ \hline
    18 & 2730 & 79 & 400 \\ \hline
    19 & 2618 & 79 & 400 \\ \hline
    20 & 4421 & 79 & 400 \\ \hline
    21 & 4113 & 79 & 600 \\ \hline
    22 & 3746 & 79 & 600 \\ \hline
    23 & 3532 & 79 & 600 \\ \hline
    24 & 3825 & 79 & 600 \\ \hline
    25 & 1096 & 79 & 200 \\ \hline
    26 & 761 & 99 & 200 \\ \hline
    27 & 2088 & 99 & 200 \\ \hline
    28 & 820 & 99 & 200 \\ \hline
    29 & 2114 & 99 & 400 \\ \hline
    30 & 1882 & 99 & 400 \\ \hline
    31 & 2159 & 99 & 400 \\ \hline
    32 & 1602 & 99 & 400 \\ \hline
    33 & 3354 & 99 & 600 \\ \hline
    34 & 2927 & 99 & 600 \\ \hline
    35 & 3031 & 99 & 600 \\ \hline
    \end{tabular}
\end{table}

Nuestro deseo es predecir las ventas del shampoo
\section{Problema de la relacion entre las horas de estudio y las clificaciones}

Un profesor desea estudiar la relacion entre las horas de estudio de un estudiante y la nota que obtiene. El profesor
realiza una encuesta a sus estudiantes preguntando cuantas horas estudia, y realiza una tabla con sus notas obtenidas 
en el parcial como sigue en la siguiente tabla:
\begin{table}[h!]
    \centering
    \begin{tabular}{|c|c|c|}
    \hline
    \textbf{Estudiante} & \textbf{Horas de Estudio (\(x\))} & \textbf{Calificación (\(y\))} \\
    1  & 1  & 10   \\ \hline 
    2  & 2  & 25   \\ \hline
    3  & 3  & 40   \\ \hline
    4  & 4  & 55   \\ \hline
    5  & 5  & 70   \\ \hline
    6  & 6  & 85   \\ \hline
    7  & 7  & 95   \\ \hline
    8  & 8  & 98   \\ \hline
    9  & 9  & 99   \\ \hline
    10 & 10 & 98  \\ \hline
    11  & 7  & 95   \\ \hline
    12  & 8  & 98   \\ \hline
    13  & 9  & 99   \\ \hline
    14 & 10 & 100  \\ \hline
    \end{tabular}
    \caption{Datos simulados de horas de estudio y calificación con tendencia a estabilización.}
    \label{tabla:horas_estudio_aplanada}
\end{table}

Se sospecha que el modelo que mejor aproxima la nube de datos de la tabla anterior es la ecuacion

\begin{center}
    $y = \alpha /x + \beta$
\end{center}

reemplazando los datos de la tabla ecuacion formamos el sistema de ecuaciones $X \vec{b} = \vec{y}$. Donde $X$ es la matriz que cumple ser:

$ X = \begin{pmatrix} 1 & 1/x_1 \\ 1 & 1/x_2 \\ ... \\ 1 & 1/x_14 \end{pmatrix}$

y $\vec{b} = (A,B)$ como $\vec{y} = (y_1, y_2,...,y_14)$ son los vectores que satisfacen el sistema de ecuaciones anterior sustituyendo los datos
de la tabla. 

Nuestro siguiente paso sera solucionar el sistema, para ello debemos multiplicar por $X^T$ miembro a miembro de la ecuacion a la izquierda,
entonces: 

\begin{center}
    $X^T X \vec{b} = X^T \vec{y}$    
\end{center}

\end{document}