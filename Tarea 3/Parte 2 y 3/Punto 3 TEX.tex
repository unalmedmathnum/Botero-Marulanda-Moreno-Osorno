\documentclass{article}
\usepackage{graphicx}
\usepackage{amsthm}
\usepackage{amsmath}
\usepackage{amssymb}
\usepackage[mathscr]{euscript}
\usepackage{marvosym}
\usepackage{titlesec}
\usepackage{enumitem}
\usepackage{bm}
\renewcommand\qedsymbol{\Squarepipe}

\titleformat{\section}
    {\bfseries\little}
    {\thetitle.\space}     
    {}                
    {}

\begin{document}

Traigamos de vuelta la ecuación diferencial $(4)$ del documento base 'Sloshing in coffee as a pumped pendulum':

\[
u''+(1+\epsilon\lambda\Omega^{2}\cos{\Omega\tau})(u-\frac{\epsilon^2u^3}{6})=0 \hspace{1cm}(4)
\]

Dado que $u$ es la posición angular del péndulo y $u'$ la velocidad angular, consideraremos el caso en el que partimos desde el punto de 0 radianes con una velocidad inicial de 0.1 (pues $u'$ también es adimensional). Así, obtenemos el siguiente PVI.

\[
\begin{cases}
    u''+(1+\epsilon\lambda\Omega^{2}\cos{\Omega\tau})(u-\frac{\epsilon^2u^3}{6})=0\\ \\
    u(0)=0\hspace{0.3cm}u'(0)=0.1
\end{cases}
\]

Llevemos esta ecuación diferencial de segundo orden a 2 ecuaciones diferenciales de orden 1.\\
\\
Sea $v_1=u$, $v_2=u'$ y $v_3=u''$, de esta forma, $v_1'=v_2$ y $v_2'=v_3=u''$, y por tanto $v_2'=(1+\epsilon\lambda\Omega^{2}\cos{\Omega\tau})(\frac{\epsilon^2v_1^3}{6}-v_1)$, así, nuestro PVI es equivalente a:

\[
\begin{cases}
v_1'=v_2\\
v_2'=(1+\epsilon\lambda\Omega^{2}\cos{\Omega\tau})(\frac{\epsilon^2v_1^3}{6}-v_1)\\ \\
\begin{bmatrix}
    v_1\\
    v_2
\end{bmatrix}
(0)=
\begin{bmatrix}
    0\\
    0.1
\end{bmatrix}
\end{cases}
\]
\end{document}
