\documentclass{article}
\usepackage{graphicx} % Required for inserting images

\title{Detalles matematicos del codigo}

\begin{document}
Consideremos la ecuacion presentada en el documento (4): 
\begin{center}
    $u''+(1+\epsilon \lambda \omega^2 cos(\omega \tau))(u - \frac{\epsilon^2 u^3}{6})$
\end{center}
donde
\begin{center}
    $\tau = wt$
    $w_0 = \frac{g}{r_0}$
    $\epsilon \lambda = \frac{\delta z}{r_0}$
    $\omega = \frac{w}{w_0}$
\end{center}
Despues de realizar con detalles los pasos para convertir la ED (4) en un sistema de EDOs, visto previamente, vamos
a utilizar el metodo de Euler mejorado (Metodo de Heun) para aproximar las soluciones de dicho sistema y la ED (4).

Como vimos previamente la ED anterior se puede llevar a un sistema de EDOs, el cual es el siguiente:

\begin{center}
        $\[
        \left\{
        \begin{array}{rcl}
        v'_1  &=& v_2 & (1) \\ 
        v'_2   &=& -(1+\epsilon \lambda \omega^2 cos(\omega \tau))(v_1 - \frac{\epsilon^2 v^3_1}{6}) & (2)
        \end{array}
        \right.
        \]$
\end{center}

Donde $v_1 = u$ y $v_2 = u'$.

Iniciamos nuestro programa definiendo las variables que haran de nuestro conjunto de datos iniciales, las cuales
sera $\epsilon, \lambda \omega \h$, un tiempo inicial, un tiempo final y una constante de frencuencia angular $w_0$.
Seguimos con crear dos funciones (f y g) que haran de (1) y (2) y que tendran como entradas variables de este mismo tipo que
luego seran modificadas por dos sucesiones:

\begin{center}
    $\[
    \left\{
    \begin{array}{rcl}
    u^*  &=& u_n + h f(t_n,u_n,v_n)\\ 
    v*   &=& v_n + h g(t_n,u_n,v_n)
    \end{array}
    \right.
    \]$
\end{center}

\end{document}