\documentclass{article}
\usepackage{graphicx} % Required for inserting images

\title{Detalles matematicos del codigo}

\begin{document}
Consideremos la ecuacion presentada en el documento (4): 
\begin{center}
    $u''+(1+\epsilon \lambda \omega^2 cos(\omega \tau))(u - \frac{\epsilon^2 u^3}{6})$
\end{center}
donde
\begin{center}
    $\tau = wt$
    $w_0 = \frac{g}{r_0}$
    $\epsilon \lambda = \frac{\delta z}{r_0}$
    $\omega = \frac{w}{w_0}$
\end{center}
Despues de realizar con detalles los pasos para convertir la ED (4) en un sistema de EDOs, visto previamente, vamos
a utilizar el metodo de Euler mejorado (Metodo de Heun) para aproximar las soluciones de dicho sistema y la ED (4).

Como vimos previamente la ED anterior se puede llevar a un sistema de EDOs, el cual es el siguiente:

\begin{center}
        $\[
        \left\{
        \begin{array}{rcl}
        v'_1  &=& v_2 & (1) \\ 
        v'_2   &=& -(1+\epsilon \lambda \omega^2 cos(\omega \tau))(v_1 - \frac{\epsilon^2 v^3_1}{6}) & (2)
        \end{array}
        \right.
        \]$
\end{center}

Donde $v_1 = u$ y $v_2 = u'$.

Entonces, ahora iniciamos nuestro programa definiendo las variables que haran de nuestro conjunto de datos iniciales, las cuales
sera $\epsilon = 0.5, \lambda = 1.0, \Omega = 1.96, h$, un tiempo inicial, 0 en nuestro contexto, un tiempo final (Como se va perturbando la posicion angular y la velocidad hasta t_parada tiempo) y una constante de frencuencia angular $w_0$.
Tomaremos de como valores iniciales (0,0.1), posicion angular en cero y velocidad 0.1.
Seguimos con crear dos funciones (f y g) que haran de (1) y (2) y que tendran como entradas variables de este mismo tipo.

Sabemos que el metodo de Heun para sistema de EDOs inicia modificando dos sucesiones: 

\begin{center}
    $\[
    \left\{
    \begin{array}{rcl}
    u^*  &=& u_n + h f(t_n,u_n,v_n)\\ 
    v*   &=& v_n + h g(t_n,u_n,v_n)
    \end{array}
    \right.
    \]$
\end{center}

las cuales en nuestro contexto son las aproximaciones a la posicion y la velocidad angular atravez del tiempi t_n con nuestros datos 
iniciales. Llamaremos a estas sucesiones los pasos predictores que van a ir corrigiendo a su vez (siguiendo la idea del mismo metodo)
a las sucesiones

\begin{center}
    $\[
    \left\{
    \begin{array}{rcl}
    u_{n+1}  &=& u_n +  \frac{h}{2}(f(t_n,u_n,v_n)+f(t_n + h,u^*,v^*))\\ 
    v_{n+1}  &=& v_n +  \frac{h}{2}(g(t_n,u_n,v_n)+g(t_n + h,u^*,v^*))
    \end{array}
    \right.
    \]$
\end{center}

En el codigo para que esto funcione de manera iterativa, creamos un ciclo While el cual tiene como variable centinela 
nuestro tiempi $t_{parada} = 50.0$, el cual ira ejecuantando estas dos sucesiones y guardando los datos en un vector,
para hacer un seguimiento de los cambios.
\end{document}